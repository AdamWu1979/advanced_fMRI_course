\documentclass[a4paper,10pt]{article}
%\documentclass[a4paper,10pt]{scrartcl}

\usepackage[utf8]{inputenc}
\usepackage[fleqn]{amsmath}
\usepackage{graphicx}
% \usepackage{wrapfig}
\usepackage{rotating}
\usepackage{lscape}

\usepackage[pdftex=true,
hyperindex=true,
colorlinks=true,pdfstartview={Fit}]{hyperref}
\hypersetup{urlcolor=blue}

\title{Advanced brain imaging methods: Tutorial 0}
\author{}
\date{3\textsuperscript{rd} February 2014}

\pdfinfo{%
/Title (Advanced brain imaging methods: Tutorial 0)
/Author ()
/Creator ()
/Producer ()
/Subject ()
/Keywords ()
}

\begin{document}
\maketitle

\section{Linear algebra: warm up}

\subsection{Adding noise to an image}

Say that for some reason you want to add normally distributed noise to an 
image. Given an image defined by the matrix A and the matrix B containing 
normally distributed noise, what would be the matrix C of the image to which 
this noise has been added ? 

\bigskip
$ A =
\left(\begin{array}{cccc}
136	& 37	& 45	& 158\\ 
28	& 69	& 78	& 55 \\
227	& 75	& 196	& 170\\
212	& 34	& 23	& 125\\
\end{array}\right) 
$
;
$B =
\left(\begin{array}{cccc}
-1	& 6	& 2	& 12\\
-14	& -6	& -4	& -1\\
-3	& 1	& 3	& -15\\
-13	& -4	& 5	& -6\\
\end{array}\right) 
$
\bigskip

\subsection{Rigid body transformation: translations}

The transformation matrix of an image is the matrix that allows you to say what 
are the coordinates [x,y,z] in world space of the voxel with indices [i,j,k] in 
voxel space.

What is the matrix B that will translate an image by 15 mm along the x 
dimension, -58 mm along the y dim and 65 mm along the z dimension?

Given the transformation matrix A below, what would be the values of the new 
transformation matrix once the translation performed by B has been applied?

\bigskip
$ A =
\left(\begin{array}{cccc}
  5 	& 0 	& 0 	& 125\\ 
  0 	& 2 	& 0 	& 118\\
  0 	& 0 	& 1 	& 10\\
  0 	& 0 	& 0 	& 1\\
\end{array}\right) 
$
\bigskip

Given this new transformation matrix, what would be the real world 
coordinates of the voxel with indices [10,25,45] and for any voxel with 
indices [i,j,k]?

\subsection{Rigid body transformation: rotations}

What is the matrix C that will rotate an image by 90 degrees around the x axis 
(pitch)? 

Given the transformation matrix A below, what would be the values of the new 
transformation matrix after this rotation has been applied?

\bigskip
$ A =
\left(\begin{array}{cccc}
  5 	& 0 	& 0 	& 125\\ 
  0 	& 2 	& 0 	& 118\\
  0 	& 0 	& 1 	& 10\\
  0 	& 0 	& 0 	& 1\\
\end{array}\right) 
$
\bigskip

\subsection{Rigid body transformation: order of transformation}

For the point with coordinates [14,58,23], what would be its coordinates if:
\begin{enumerate}
  \item you applied the translation defined by B then the rotation defined by C,
  \item you applied the rotation defined by C and then the translation defined 
by B?
\end{enumerate}


\section{General linear model}

In class we have seen how to find the equation that defines the ordinary 
least-square estimate of the $\beta$ value of the following model:
\begin{displaymath}
  y = \beta x + \varepsilon
\end{displaymath}

For each value of $x_{i}$, the residual error ($\varepsilon_{i}$) is given by the difference between the predicted value of the model ($\hat{y_{i}} = \beta x_{i}$) and the actual empirical value ($y_{i}$). We want to minimize the sum of the squared residual:
\begin{displaymath}
  \sum_{i=1}^{N} \varepsilon_{i}^{2} = \sum_{i=1}^{N}(y_{i} - \hat{y_{i}})^{2} = \sum_{i=1}^{N} (y_{i} - \beta x_{i})^{2} = \sum_{i=1}^{N} (y_{i}^{2} - 2\beta x_{i}y_{i} + \beta^{2} x_{i}^{2})
\end{displaymath}
At the value of $\beta$ that minimize this function, we know that its 
derivative with respect to $\beta$ will be equal to 0. So to find $\beta$, 
we derived this function with respect to $\beta$, and rearranged this 
derivative to have the value of $\beta$ expressed in terms of $x$ and $y$.

\medskip
Let's now try to do the same with this model:
\begin{displaymath}
  y = \beta_{1} x + \beta_{0} + \varepsilon
\end{displaymath}

Do this by the following steps:
\begin{enumerate}
 \item Write down the sum of the squared residual of this model.
 \item Derive this function with respect to $\beta_{1}$ and rearranged this 
derivative to have the value of $\beta_{1}$ expressed in terms of $x$, $y$ and  
$\beta_{0}$.
 \item Do the same with respect to $\beta_{0}$.
 \item Substitute $\beta_{0}$ in the function found in step 2 by the solution 
of step 3.
\end{enumerate}


\section{Design matrices}

The General Linear Model described by $Y=X\beta+\varepsilon$. Define the design matrices (X) of the following:
\begin{enumerate}
  \item a two-sample t-test with 2 subjects in group A and 3 subjects in B,
  \item a paired t-test with 5 subjects with two conditions a and b,
  \item an ANOVA with three groups of subjects and 3 subjects in each group,
  \item a repeated-measures ANOVA for 3 subjects and with three within-subject levels (a, b, c).
\end{enumerate}

Here are some advices:
\begin{itemize}
 \item Do not panic!
 \item For 1: 
  \begin{enumerate}
    \item write for each subject the equation that describes that subject's $y$ in terms of:
      \begin{itemize}
	\item the average of the group A ($\mu_{A}$) times a certain weighting factor, 
	\item the average of the group B ($\mu_{B}$) times a certain weighting factor,
	\item an error term $\varepsilon$ specific to that subject.
      \end{itemize}
    \item go from the equation form to the matrix form, by putting together a) the $y$ of all subjects, b) the weighting factors, c) the 2 averages and d) the error terms.
    \item the design matrix is the one that contains the weighting factors.
  \end{enumerate}
  
 \item For 2, the process will the same as for 1 but you will have to describe each subject's $y$ in each condition in terms of:
    \begin{itemize}
      \item the average of the condition a ($\mu_{a}$) times a certain weighting factor, 
      \item the average of the condition b ($\mu_{b}$) times a certain weighting factor,
      \item as many subject specific variable ($\tau$) as there are subjects each time multiplied by a certain weighting factor,
      \item an error term specific to that subject and that condition.
    \end{itemize}

    \item For 3 and 4, you can geneleralize by remembering that a two sample t-test is in fact an ANOVA with only two groups and that a paired t-test is in fact a repeated-measures ANOVA with only two levels.
\end{itemize}

\end{document}
