\documentclass[a4paper,10pt]{article}
%\documentclass[a4paper,10pt]{scrartcl}

\usepackage[utf8]{inputenc}
\usepackage[fleqn]{amsmath}
\usepackage{graphicx}
\usepackage{rotating}
\usepackage{lscape}

\usepackage[pdftex=true,
hyperindex=true,
colorlinks=true,pdfstartview={Fit}]{hyperref}
\hypersetup{urlcolor=blue}

\title{Advanced brain imaging methods: Tutorial 1}
\author{}
\date{20\textsuperscript{th} January 2014}

\pdfinfo{%
/Title (Advanced brain imaging methods: Tutorial 1)
/Author ()
/Creator ()
/Producer ()
/Subject ()
/Keywords ()
}

\begin{document}
\maketitle



\section{Preliminary remarks}

MRI images are composed of 2 parts:
\begin{enumerate}
  \item a header that contains the information about the image (transformation matrix, subject information, scanner and sequences parameters and other types of meta data),
  \item an image that can be a 3D matrix that contains one volume (an anatomical scan for example) or a 4D matrix for a time series of volume (a series of EPI volumes acquired during one fMRI session).
\end{enumerate}

There are quite a few different formats of images. SPM8 can easily deal with images in an Analyze format where header and image are stored in 2 different files (with a .hdr and a .img extension respectively). But the field tends more and more to use the Nifti format where header and image are stored in one single .nii file. Scanners might often output images under a DICOM format that needs to be converted: in SPM you can use the \verb* DICOM_import  tool (\textit{Batch $\rightarrow$ SPM $\rightarrow$ Util $\rightarrow$ DICOM import}).



\section{Reading header information}

This tutorial is based on SPM, so make sure that you have downloaded the software from \href{http://www.fil.ion.ucl.ac.uk/spm/software/download.html}{there}. We will be using the \href{http://www.fil.ion.ucl.ac.uk/spm/data/auditory/}{single subject epoch (block) auditory fMRI activation data set} from the SPM website, so download it as well. I also strongly advise you to do the preprocessing and the first level analysis related to this data set. Everything is explained in the chapter 28 of the \href{http://www.fil.ion.ucl.ac.uk/spm/doc/spm8_manual.pdf}{SPM manual} (for the time being, you do not have to worry about the Bayesian analysis part).

\bigskip

SPM uses the function \verb* ImgHeader=spm_vol('ImageName.img')  to read information contained in the header of an image and store in a structure in the variable \verb* ImgHeader . 

A matlab structure has several fields (usually with pretty transparent names), each associated with a value. For example, the value associated to the field 'dim' can be read with \verb* ImgHeader.dim  contains the dimension of the image associated to this header. You can also directly get the value transformation matrix by using the function \verb* spm_get_space(ImageName) .

Using either of these methods, get the header information of the non-normalized structural scan from the data set you downloaded from the SPM website. From this header can you tell:
\begin{enumerate}
  \item How many slices do you have in this image? What are the dimensions (the number of voxels) of each slice?
  \item What is the transformation matrix of this image (i.e what are its values in each row and column)? 
\end{enumerate}

From this matrix:
\begin{enumerate}
  \item How can you tell what are the dimensions of each voxel (in mm) of this structural scan?
  \item How can you tell the real world coordinates (in mm) of the voxel with the indices [1,1,1]? 
  \item What matlab code would you use to generalize this to any coordinates in mm (i.e get the real world coordinates of a given [x,y,z] voxel)? This will require to multiply 2 matrices (check the slides of the pre-processing lecture to know which one).
  \item How could you tell the [x,y,z] indices of the voxel that contains the origin of the image, i.e with world space coordinates of [0,0,0] mm? 
  \item What matlab code would you use to generalize this to any voxel (i.e get the voxel space coordinates a point with some given world space coordinates)?
\end{enumerate}

\medskip
\textit{Hints:}
\begin{itemize}
 \item Remember that if a matrix A perfoms a certain transformation, you can find the matrix that performs the \textquoteleft reverse\textquoteright transformation, by finding the inverse of A (check the Maths4SPM.pdf in additional material for a reminder of what the inverse of a matrix is). 
 \item For those questions, you can easily check you answers or your code by using the \textbf{Display} button of the graphic interface of SPM. You can also use the \textbf{Check Reg} button and right-clicking on the image to get some of those informations. 
 \item Questions 3 and 5 will require you to do some matrix multiplication, so remember that there are restrictions (in terms of dimension) on how to do those multiplications: you might need to add dummy columns or rows to do some of those multiplications.
\end{itemize}

\textit{Matlab tips:} 
\begin{itemize}
 \item If you want help on how to use a certain function, just type in the command line \verb* help  followed by the function name. 
 \item To create a row vector \verb* VoxCoord  containing, for example, the voxel coordinates $(125, 156, 25)$, you should type \verb* VoxCoord=[125,156,25]  . If you wanted to have the same but as a column vector, you would type \verb* VoxCoord=[125;156;25]  .
 \item To add values to a row vector, you could do \verb* VoxCoord=[VoxCoord,25]  and similarly for a column vector \verb* VoxCoord=[VoxCoord;25]  .
 \item You can transform a row vector into a column vector, by taking its transpose: \verb* VoxCoord=trans(VoxCoord)  or more simply \verb* VoxCoord=VoxCoord'  .
 \item The command \verb* inv(A)  will give you the inverse of the matrix A.
\end{itemize}



\section{Reading values from an image}

This code will give you the content of an image and store in a variable \verb* ImgContent .
\begin{verbatim}
  ImgHeader=spm_vol('ImageName.img')
  ImgContent=spm_read_vols(ImgHeader)
\end{verbatim}

\textit{Matlab tips:} 
\begin{itemize}
 \item If you want to suppress the output to the screen (i.e make sure that matlab will not display the result of a certain command or calculation) add an \verb* ;  at the end of the line,
 \item If you want to stop an on-going action matlab is doing (displaying some results or doing some computation), press ctrl+C.
\end{itemize}

If you wanted to know the intensity in the voxel with coordinates $(125, 156, 25)$ you could do either this \verb* ImgContent(125,156,25)  or that: 
\begin{verbatim}
  VoxCoord=[125 156 25]
  ImgContent(VoxCoord(1),VoxCoord(2),VoxCoord(3))
\end{verbatim}

Similarly if you wanted to have all the values of the 25\textsuperscript{th} slice in the z direction (axial), you would type \verb* ImgContent(:,:,25)  .

\bigskip
On the first EPI image, find a voxel that is roughly in the left auditory cortex (by using \textbf{Display} or \textbf{Check Reg}) and using the information above give the signal intensity found in this voxel.



\section{Reading a time-series} \label{Reading a time-series} 

Now try to plot (using the function \verb* plot  ) the time-series of signal intensities of that voxel across the all the realigned EPI volumes: i.e for each EPI acquired in this experiment, what is the BOLD signal value in this voxel? 

First, if you do not know where the realigned images are, I suggest you try to run the \textbf{Realign: estimate and write} button on the raw EPI images. 

Once this is done, I suggest that you make sure that the world space origin of those images is situated at the anterior commissure. To do this, use the \textbf{Check Reg} button to open the mean EPI image created by the realignement and check if the origin of this image is at the anterior commissure (if you have no idea where the anterior commissure is check \href{http://headneckbrainspine.com/web_flash/newmodules/Brain\%20MRI.swf}{here}). If you have to change the origin of the image: position the crosshair where you want the new origin to be, right-click on the image, select \textit{Reorient image(s)} and then select \textit{Set origin to Xhairs}. When asked, select all the realigned and unwarped images (usually with a 'u' prefix) to apply this change of origin to all the fuctionnal images.

To do the rest of this task, you can either use loop to open each image, get the value at the desired coordinates and then open the next image, like this:
\begin{verbatim}
  for ImageIndex = 1:TotalNumberOfImages
    % Get image header
    ImgHeader=spm_vol(ImageList(ImageIndex));
    
    % Get image content
    ImgContent=spm_read_vols(ImgHeader);
    
    % Get signal value at coordinate i,j,k
    SignalValue=ImgContent(i,j,k);
    
    % Add it at the right position in the time series of values 
    TimeSeries(ImageIndex)=SignalValue; 
  end
\end{verbatim}

Alternatively, you can use the fact that \verb* spm_vol  can accept a matrix of filenames (see below) and open all the files at once in one 4D matrix:

\begin{verbatim}
  ImgsMat = ['file_name_1.img';
	    'file_name_2.img';        
	    'file_name_3.img'];
      
  ImgHeaders = spm_vol(imgsMat);
  
  ImgContents=spm_read_vols(ImgHeader);
\end{verbatim}

\medskip
\textit{Hints:} 
\begin{itemize}
 \item To get the list of images in a folder, you can either use \verb* ImageList=ls('*.img')  or \verb* ImageList=dir('*.img')  to store this list in a structure in the variable \verb* ImageList  . The name of each file will be store in the field 'name' of that structure, that you can read like this for the first image: \verb* ImageList(1).name  .
 \item An easier, but less scriptable, way to do it is to use the following spm function \verb* ImageList=spm_select(Inf,'image')  to have a matrix with all the absolute images names.
\end{itemize}



\section{Estimating misalignment using different cost functions}

As we have seen in class, estimating how misaligned two images are is done using cost functions. 

The cost function typically used by SPM during realignment is the least-squares cost function:
\begin{displaymath}
 C = \sum_{v=1}^{N} (A_{v} - B_{v})^2
\end{displaymath}

This function computes the sum, from the voxel ($v$) 1 to N, of the square of the difference in intensities between the image A and B.

\medskip 
FSL uses the normalized correlation for its realignment:
\begin{displaymath}
 C = \frac{\sum_{v=1}^{N} (A_{v}B_{v})}{\sqrt{\sum_{v=1}^{N} A_{v}^2} \sqrt{\sum_{v=1}^{N} B_{v}^2}} 
\end{displaymath}

As the misalignment between the 2 images diminishes, which value should each of these cost functions tend towards? 

Get the data from the first \textit{\textbf{raw}} EPI image and store it in a variable called \verb* Volume_1 . Now create the following variable :
\begin{itemize}
  \item \verb* Volume_1_offset  where all the voxel intensities of \verb Volume_1  will be offseted by +10\% of the mean value of this volume,
  \item \verb* Volume_1_noise  where you will have added Gaussian random noise to \verb Volume_1  ,
  \item \verb* Volume_1_invert  where voxels with a high intensity in \verb Volume_1  will now have a low intensity and vice versa.
\end{itemize}

Also get the data from the second raw EPI image and the second realigned and resliced EPI image and store their values in variables called \verb* Volume_2 and \verb* RealignedVolume_2  .

Estimate the misalignment between \verb Volume_1  with itself and with each of the 5 other images using the normalized correlation and the least-squares cost functions. Which comparison illustrates best the fact that these cost functions are not appropriate for inter-modality coregistration?

\textit{Matlab tips:} 
\begin{itemize}
 \item you can add Gaussian random noise to an image of dimension (X, Y, Z) by adding to your original image, an image of Gaussian noise of same dimension created with the command \verb* Noise=randn(X,Y,Z,)*sigma+mu  where \verb* sigma  and \verb* mu  are the variables that contain the values of standart deviation and mean value you want to your Gaussian noise to have. For this exercice you can try different values for both parameters.
 \item if you want to perform some operation on the element of a matrices and not on the matrices themselves, you have to use a  \verb* .  in front of the operation sign you are perfoming. For example, say you want to multiply the value of each voxel of the image A with its corresponding value of the image B, you should then type \verb* A.*B  because the operation \verb* A*B  would multiply the whole matrice A by the matrice B.
 \item Usual mathematical functions in matlab that you might need: \verb* sum , \verb* mean , \verb* sqrt , \verb* ^ . 
\end{itemize}



\section{Generating coregistration histograms}

Ensure that the origin of the structural is located at the anterior commissure. If it is not, change it as explained in the section \ref{Reading a time-series}.

Once this is done, coregister the structural image to the mean EPI image outputted by the realignment of the EPIs: use the \textbf{Coregister: estimate} button. At the end of this processing step, SPM will display a mutual coregistration histogram. Intermodality coregistration uses methods that try to minimize the entropy (the disorder) on these histograms: it is sometimes quite obvious when one compare the before and after histograms.

For two images of same dimensions, you can make a similar looking histogram by creating a scatter plot in matlab using the \verb* scatter  function with as many dots in the scatter plot as there are voxels in the image and where the x coordinate of the $i^{th}$ dot is given by intensity in the $i^{th}$ voxel of the first image and the y coordinate of the $i^(th)$ dot is given by intensity in the $i^{th}$ voxel of the second image:
\begin{verbatim}
  scatter(Image_A(:), Image_B(:));
\end{verbatim}

What should the histogram look like when the two images are the same? And when one image is the 'negative' of the other? 

Check this by plotting the histograms that compares the \verb Volume_1 with each of the 5 other images of the precedent question. Why is it better to have to rely on those histograms to evaluate the coregsitration of two images from different modality ?



\section{From native space to MNI space and back}

The file Left\_A1.nii and Right\_A1.nii contain the masks for the left and right auditory cortices in MNI space. Voxels in the masks have a value of 1 when this voxel is within the region of interest (ROI) and a value of 0 when it is outside of this ROI. Use the \textbf{Check Reg} button to open the non-normalized structural scan and the two masks. Since the masks are in MNI space and the structural scan is still in native space (in the coordinate referential of the scanner), we need to transform those masks from MNI space to native space. To do this you need :
\begin{enumerate}
  \item to get the transformation parameters that would allow you to normalize the structural from its native space to MNI space or the transformation parameters that will do the the inverse normalization. To get those parameters segment the structural image using the  SPM \textbf{Segment} button,
  \item 'inverse-normalize' the masks using the SPM \textbf{Normalize: write} button with: 
    \begin{itemize}
      \item seg\_inv\_sn.mat file outputted by the segmentation which is the file that contains parameters that will transform an image from MNI space to native space,
      \item the following expression for the bounding box value: ones(2,3)*NaN,
      \item the following values for the voxel sizes: [NaN NaN NaN].
    \end{itemize}
\end{enumerate}



\section{Reading values within a region of interest}
With those new masks, you are now going to create a figure showing how the average BOLD signal in each auditory cortices accross evolves accross the time series of realigned EPI images.
\begin{itemize}
  \item First use the \textbf{Coregister: reslice} function to make sure that the masks have the same resolution as the mean EPI image (use a nearest neighbour interpolation),
  \item You can now modify the code you created for the section \ref{Reading a time-series} so that for every volume of the time series, you compute the mean of the values that are within the ROI defined by the mask with the following code \verb* mean(Volume_Data(Mask_Data)) .
\end{itemize}



\section{Displaying sections of an image}

Write the code that will display on the same figure the sagital, coronal and axial views of the structural scan for a given set of [x,y,z] coordinates in world space?

To help you, here are the different steps of this code:
\begin{itemize}
 \item Provide the file name of the image and the world space coordinates that should be displayed.
 \item Get the header information of that image.
 \item Get the content of that image.
 \item Get the transformation matrix of that image.
 \item Transform the world space coordinates into voxel space coordinates. For simplicity round those new coordinates using the function \verb* round .
 \item Use the \verb* imagesc  function to display sagital, coronal and axial in 3 different subplots (\verb* subplot  function) of the same figure. The function \verb* squeeze  might also come in handy here.
\end{itemize}



\section{Applying affine transformations to an image}

As its name suggests, the function \verb* spm_write_vol(ImgHeader,ImgContent) will create an image given some data and header information. For example, the following code will open an image and create a copy of it:
\begin{verbatim}
  % Get the header
  ImgHeader = spm_vol('filename.nii');
  % Get the data associated
  Data = spm_read_vols(ImgHeader);

  % Creates a copy fo the header
  NewImgHeader = imgInfo;

  % Change the name in the header
  NewImgHeader.fname = 'copy_of_filename.nii';
  NewImgHeader.private.dat.fname = NewImgHeader.fname;

  % Write the copy
  spm_write_vol(NewImgHeader,Data);
\end{verbatim}

The function \verb* spm_matrix(P,'X')  returns an affine transformation matrix given a certain vector P of parameters:
\begin{itemize}
  \item P(1)  - x translation
  \item P(2)  - y translation
  \item P(3)  - z translation
  \item P(4)  - x rotation about - {pitch} (radians)
  \item P(5)  - y rotation about - {roll}  (radians)
  \item P(6)  - z rotation about - {yaw}   (radians)
  \item P(7)  - x scaling
  \item P(8)  - y scaling
  \item P(9)  - z scaling
  \item P(10) - x affine
  \item P(11) - y affine
  \item P(12) - z affine
\end{itemize}

\bigskip
Write the codes that will open the structural image do the following:
\begin{itemize}
 \item move the image by by 15 mm along the x dimension, -58 mm along the y dim and 65 mm along the z dimension and then save this new image,
 \item rotate an image by 90 degrees around the x axis (pitch) and then save this new image,
 \item applies the translation, then the rotation and then save this new image, 
 \item applies the rotation, then the translation and then save this new image.
\end{itemize}

Once you have done that use the \textbf{Check reg} button to open the 4 images you have created and compare them to the original.


%\section{Slice timing correction}
%with linear interpolation
%with sync interpolation

%\section{Smoothing}

% \section{Estimating misalignment using mutual information}


\end{document}
